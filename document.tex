\documentclass{bioinfo}
\copyrightyear{2015}
\pubyear{2015}
\usepackage{listings}
\usepackage{color}
\definecolor{lightgray}{rgb}{.9,.9,.9}
\definecolor{darkgray}{rgb}{.4,.4,.4}
\definecolor{purple}{rgb}{0.65, 0.12, 0.82}

\lstdefinelanguage{JavaScript}{
  keywords={typeof, new, true, false, catch, function, return, null, catch, switch, var, if, in, while, do, else, case, break},
  keywordstyle=\color{blue}\bfseries,
  ndkeywords={class, export, boolean, throw, implements, import, this},
  ndkeywordstyle=\color{darkgray}\bfseries,
  identifierstyle=\color{black},
  sensitive=false,
  comment=[l]{//},
  morecomment=[s]{/*}{*/},
  commentstyle=\color{purple}\ttfamily,
  stringstyle=\color{red}\ttfamily,
  morestring=[b]',
  morestring=[b]"
}

\lstset{
   language=JavaScript,
   backgroundcolor=\color{lightgray},
   extendedchars=true,
   basicstyle=\footnotesize\ttfamily,
   showstringspaces=false,
   showspaces=false,
   numbers=left,
   numberstyle=\footnotesize,
   numbersep=9pt,
   tabsize=2,
   breaklines=true,
   showtabs=false,
   captionpos=b
}


\begin{document}
\firstpage{1}

\title[short Title]{ECMAScript 6 - Language Features}
\author[Sample \textit{et~al}]{Corresponding Author\,$^{1,*}$, Co-Author\,$^{2}$ and Co-Author\,$^2$\footnote{to whom correspondence should be addressed}}
\address{$^{1}$Department of XXXXXXX, Address XXXX etc.\\
$^{2}$Department of XXXXXXXX, Address XXXX etc.}

\history{Received on XXXXX; revised on XXXXX; accepted on XXXXX}

\editor{Associate Editor: XXXXXXX}

\maketitle

\begin{abstract}

\section{Motivation:}
ECMAScript is the underlying standard for a cross-platform programming languague usually embedded in HTML Code to add dynamic interactivity to web documents. 
Almost every Web Developer has heard about this wide spread Language and came to the point using the widely known ECMAScript Implementation "Javscript" for her or his project. 
As time passes by, the use of JavaScript in practice has shown it's benefits and weaknesses.
That makes it important to stay up to date with the current and upcoming language standards for this very potent programming language.  
In this article we're going to have a closer look at the upcoming JavaScript language standard which is ECMAScript Version 6.
After making clear 

\section{Contact:} \href{name@bio.com}{name@bio.com}
\end{abstract}
\section{Introduction}
\subsection{ECMAScript \& JavaScript}
ECMAScript is the language specification for JavaScript standartized by ECMA International. The corresponding standard is ECMA-262 specification. As ECMAScript is also standartized
by the International Organization for Standardization (ISO) in collaboration with the International Electrotechnical Commission (IEC) the corresponding standard is ISO/IEC 16262.
As JavaScript provides additional features than the ones specified in the ECMAScript language standard it can be seen as a ECMAScript variant. Another popular variant of ECMAScript is JScript.
\subsubsection{ECMA International}
ECMA International is a not-for-profit association in Geneva. 
Their target is to develop and publicate standads and technical reports for information- and
communication technology as well as consumer electronics. 
%quelle: http://www.ecma-international.org/activities/General/presentingecma.pdf
\subsubsection{Next standard for JavaScript}
The current standard for JavaScript is ECMAScript 5.1, released in 2011. 
ECMAScript 6 will be the follow-up standard for the JavaScript programming language.
Browser vendors can use it as an orientation for their own JavaScript implementations.

\subsection{History}
In 1996 Netscape submitted JavaScript to ECMA International for standartization.
\subsubsection*{\textbf{ES1 - 1997}}
First ECMAScript standard released by ECMA International, named ECMAScript Version 1.
\subsubsection*{\textbf{ES2 - 1998}}
One year after the release of ECMAScript Version 1, editorial changes were made to align ECMAScript with the ISO/IEC international standard 16262.
\subsubsection*{\textbf{ES3 - 1999}}
Modern language features were added in ECMAScript Version 3. Some examples are regular expression support or exception handling with try/catch.
\subsubsection*{\textbf{ES4 - (---)}}After the first set of changes made in ECMAScript Version 3, some members of ECMA International planned a set of breaking changes for Version 4, after a huge discussion about backwards compatibility Version 4 was finally abandoned. 
\subsubsection*{\textbf{ES5 - 2009}}
After endless debates and discussion, the plan for breaking changes was temporarily freezed and reduced to the clarification of JavaScript ambiguities with the introduction of the strict mode.
\subsubsection*{\textbf{ES5.1 - 2011}}
A second set of editorial changes in ECMAScript History were made to align it again with theISO/IEC standard 16262.
\subsubsection*{\textbf{ES6 - 2015}}
ECMAScript 6 (aka JS Harmony) is scheduled for june 2015. It will be the most extensive standard since ECMA Script Version 1 was released. Due to the fact, that ES6 just adds syntax features on top of ESCMAScript 5.1 the backward compatibility is guaranteed. The major changes made in ES6 also work as a foundation for the following features planned for ES7.
\subsubsection*{\textbf{ES7 - (tbd)}}
In May 2015 ECMAScript Version 7 is in a very early draft state, but already planned. 

\subsection{Benefits of ECMAScript6}
\subsubsection{Suitable for large code bases}
The new ECMAScript Version 6 provides modules an classes which makes the prior used patterns (e.g. module pattern) obsolete
and allows better encapsulation. Along with other new features this gives the possibility for clean easily readable code, 
which makes ES6 suitable for bigger projects.
Thanks to the syntax improvements the boilerplate can be reduced which also leads to less lines of code,
reduced maintenance effort and easier extensibility, if used properly.
\subsubsection{Straight forward migration of existing code}
As ECMAScript 6 only adds features on top of ECMAScript 5.1 the migration in existing projects is very easy,
because there's no nescessarity to rewrite existing code to make it fit the actual standard.  

\section{Web App}

\subsection{ESnake6}
To demonstrate the basics as well as advanced concepts of the ECMAScript 6 standard we created a modified version of the 
popular snake game.  

\begin{figure}[!tpb]%figure351
\centerline{\includegraphics[scale=0.35]{esnake.jpg}}
\caption{ECMAScript 6 web app: \textit{ESnake6}}\label{fig:03}
\end{figure}

\subsection{Architectural Overview}
\subsubsection{Build Process}
To build ECMAScript 6 applications while the standard is still in a draft state, we used a ES6 to ES5 compiler.
After writing the App the ECMAScript Version 6 source files will be translated into ECMAScript 5.1. while the 
used HTML and CSS files as well as the included libraries will be copied. For this action a build step is nescessary. 
After the translation into ECMAScript 5.1 the project can be run with RequireJS modules and Sourcemaps.
%first translate in a build step
%bild build process
\subsubsection{Libraries and Technologies}
The used technologies in the app include the Bower package manager, which was used to manage client side libraries. The task runner gulp automated the build system.
For the sake of testing http-server was used to serve static files. The RequireJS Module system handles asynchronous loading of javascript. The Loadash utility library
provides fuctions for working with lists (e.g. map, filter, last). Finally the Babel compiler compiles ECMAScript 6 into ECMAScript 5, which is understood by all modern browsers.
This specific compiler accepts a flag which causes the compiler to translate modules to RequireJS syntax.

\begin{figure}[!tpb]%figure351
\centerline{\includegraphics[scale=0.35]{esp.png}}
\caption{Libraries and technologies}\label{fig:01}
\end{figure}


\subsubsection{Application Architecture}
The entry point for the application is in the index.html file, which uses RequireJS to load main.js. After setting up the game 
loop using \textit{requestAnimationFrame} the game class gets instantiated which again instatiates the Grid, Snake and Food class,
as well as FoodRandom. To give an example of Inheritation in ECMAScript 6 FoodRamdom inheritates from Food and appears randomly for
a short period of time.
\begin{figure}[!tpb]%figure351
\centerline{\includegraphics[scale=0.26]{apparch.jpg}}
\caption{Applicatoin architecture}\label{fig:02}
\end{figure}
\subsubsection{Project Structure} 
The app folder contains the nescessary HTML, CSS and JavaScript (ECMAScript6) code. The application entry point is the index.html file, wich loads main.js.

\section{ECMA Script language features}

\subsection{Language basics}

\subsubsection{Const \& Let}
A simple new and useful features is the "'const"' keyword, which makes a variable to a constant, which means, 
that it can only be assigned once. The assignement of a second value will fail silently but can be reognized by
Javascript linters.
\begin{lstlisting}[caption= My Javascript Example]
const myConst = 4;
 myConst =2; // will fail silently
\end{lstlisting}

The new "'let"' keyword is meant to be a replacement for the currently used "'var"', which defines variables 
either globally or in context of the whole codeblock, ignoring block scopes. As it may lead to unexpected behaviour
it's a source for nasty bugs, which is meant to be eliminated with "'let"'. The new keyword declares variable in 
regard of block scope. Which means that the variable scope expires after leaving the code block it was declared in.

%TODO code example

\subsubsection{Destructuring}
Destructuring might remind of pattern matching known from several functional programming languages.
The feature used on arrays allows the programmer to assign different kinds of values to specific array elements.
\begin{lstlisting}[caption= My Javascript Example]
// arrays
var [a, b] = ["hello", 4];
console.log(a); // => hello
console.log(b); // => 4
\end{lstlisting}

Another example for the use of destructuring on arrays is the standard case of swapping the values of two variables.
This is as easily possible as shown in the following example.
\begin{lstlisting}[caption= My Javascript Example]
// Swap variables easily without temp
var a = 1, b = 2;
[b, a] = [a, b];
console.log(a, b); // => 2 1
\end{lstlisting}


\begin{lstlisting}[caption= My Javascript Example]
// objects
var {userId: x} = {userId: 5};
console.log(x); // => 5
\end{lstlisting}

\subsubsection{Default, Rest, Spread parameters}
Default parameters allow the programmer to predefine values for function parameters which will be automatically
set if the function is called with less parameters than defined or if the parameters value is "'undefined"'.
This feature makes it much easier to make sure function parameters got reasonable values without nescessarity to
use if blocks to do so.

\begin{lstlisting}[caption= My Javascript Example]
// Game.js
function createGame(canvas, cellsX = 20, cellsY = 20, playerCount = 4) { ¿ }

// main.js
const context = canvas.getContext("2d");
const game = createGame(canvas, 40, 40); // sets cellsX and cellsY to 40, playerCount defaults to 4
\end{lstlisting}

Rest parameters are indicated by  "..." followed by the parameter name. As you can see in the following 
example the parameters of "testRest" that follow x get stored in an array called y.

\begin{lstlisting}[caption= My Javascript Example]
// put the rest of all function arguments into an array
function testRest(x, ...y) {
          return x * y.length; // y is an Array
  }
  testRest(3, "hello", true) == 6
\end{lstlisting}

Spreading of parameters is also indicated by "..." followed by an array. 
\begin{lstlisting}[caption= My Javascript Example]
// turn array into argument list and call function
function testSpread(x, y, z) {
          return x + y + z;
  }
  testSpread(...[1,2,3]) == 6 // Pass each elem of array as argument
\end{lstlisting}

\subsubsection{Template Strings}

\begin{lstlisting}[caption= My Javascript Example]
// classes/Snake.js
alert(`${this.name} collided with ${player.name}`); // use back ticks for string interpolation

// ES5
alert(this.name + “ collided with ” + player.name);
\end{lstlisting}

\subsubsection{Arrow Functions}

\begin{lstlisting}[caption= My Javascript Example]
// Game.js
this.players.forEach(player => player.update());
// ES5
this.players.forEach(function(player) { return player.update(); });

// classes/Snake.js
draw(ctx) {
  this.positions.forEach(pos => { // ES5: function(pos) { ...
    const { x: pixPosX, y: pixPosY } = this.grid.cellPosToPixelPos(pos);

    ctx.fillStyle = this.color;
    ctx.fillRect(pixPosX, pixPosY, this.grid.cellWidth, this.grid.cellHeight);
  });
}
\end{lstlisting}

\subsubsection{Enhanced Object Literals}

\begin{lstlisting}[caption= My Javascript Example]
ivar obj = {
    // __proto__
    __proto__: theProtoObj,
    // Shorthand for ‘handler: handler’
    handler,
    // Methods
    toString() {
     // Super calls
     return "d " + super.toString();
    },
    // Computed (dynamic) property names
    [ "prop_" + (() => 42)() ]: 42
};
\end{lstlisting}
\subsection{Code organization}



%\section{Discussion}
%As it's possible to extend existing projects with ECMAScript 6 code, the question comes up if a mixture of known concepts like the Module Pattern 
%and the new Module System doesn't create confusion. 

\section{Conclusion}
\begin{itemize}
        \item ECMAScript 6 adds on top of ECMAScript 5 and thus maintains backwards compatibility which makes it easy to extend existing projects with new ECMAScript 6 code. 
        \item Many syntax improvements enable the programmer to reduce boilerplate and reduce maintenance efforts. 
        \item Thanks to Modules and Classes code organization gets a lot easier and makes ECMAScript 6 suitable for bigger projects.
\end{itemize}
To use ECMAScript 6 today, you can use an ES6 to ES5 compiler e.g. Babel or Traceur. The development experience gets
very pleasant thanks to sourcemaps. For new runtime features you can also use polyfills (e.g. core-js).

\section*{Acknowledgement}
Text Text Text Text Text Text  Text Text.  \citealp{Boffelli03} might want to know about  text text text text

\paragraph{Funding\textcolon} Text Text Text Text Text Text  Text Text.

%\bibliographystyle{natbib}
%\bibliographystyle{achemnat}
%\bibliographystyle{plainnat}
%\bibliographystyle{abbrv}
%\bibliographystyle{bioinformatics}
%
%\bibliographystyle{plain}
%
%\bibliography{Document}


\begin{thebibliography}{}
\bibitem[Bofelli {\it et~al}., 2000]{Boffelli03} Bofelli,F., Name2, Name3 (2003) Article title, {\it Journal Name}, {\bf 199}, 133-154.

\bibitem[Bag {\it et~al}., 2001]{Bag01} Bag,M., Name2, Name3 (2001) Article title, {\it Journal Name}, {\bf 99}, 33-54.

\bibitem[Yoo \textit{et~al}., 2003]{Yoo03}
Yoo,M.S. \textit{et~al}. (2003) Oxidative stress regulated genes
in nigral dopaminergic neurnol cell: correlation with the known
pathology in Parkinson's disease. \textit{Brain Res. Mol. Brain
Res.}, \textbf{110}(Suppl. 1), 76--84.

\bibitem[Lehmann, 1986]{Leh86}
Lehmann,E.L. (1986) Chapter title. \textit{Book Title}. Vol.~1, 2nd edn. Springer-Verlag, New York.

\bibitem[Crenshaw and Jones, 2003]{Cre03}
Crenshaw, B.,III, and Jones, W.B.,Jr (2003) The future of clinical
cancer management: one tumor, one chip. \textit{Bioinformatics},
doi:10.1093/bioinformatics/btn000.

\bibitem[Auhtor \textit{et~al}. (2000)]{Aut00}
Auhtor,A.B. \textit{et~al}. (2000) Chapter title. In Smith, A.C.
(ed.), \textit{Book Title}, 2nd edn. Publisher, Location, Vol. 1, pp.
???--???.

\bibitem[Bardet, 1920]{Bar20}
Bardet, G. (1920) Sur un syndrome d'obesite infantile avec
polydactylie et retinite pigmentaire (contribution a l'etude des
formes cliniques de l'obesite hypophysaire). PhD Thesis, name of
institution, Paris, France.

\end{thebibliography}
\end{document}
