\documentclass{bioinfo}
\copyrightyear{2015}
\pubyear{2015}
\begin{document}
\firstpage{1}

\title[short Title]{ECMAScript 6 - Language Features}
\author[Sample \textit{et~al}]{Corresponding Author\,$^{1,*}$, Co-Author\,$^{2}$ and Co-Author\,$^2$\footnote{to whom correspondence should be addressed}}
\address{$^{1}$Department of XXXXXXX, Address XXXX etc.\\
$^{2}$Department of XXXXXXXX, Address XXXX etc.}

\history{Received on XXXXX; revised on XXXXX; accepted on XXXXX}

\editor{Associate Editor: XXXXXXX}

\maketitle

\begin{abstract}

\section{Motivation:}
ECMAScript is the underlying standard for a programming languague usually embedded in HTML Code to add dynamic interactivity to web documents. 
Almost every Web Developer has heard about this wide spread Language and came to the point using Javscript for her or his project. 
As time passes by, the use of JavaScript in practice has shown it's benefits and weaknesses.
That makes it important to stay up to date with the current and upcoming language standards for this very potent programming language.  
In this article we're going to have a closer look at the upcoming JavaScript language standard which is ECMAScript Version 6.
After making clear 

\section{Results:}

\section{Availability:}

\section{Contact:} \href{name@bio.com}{name@bio.com}
\end{abstract}

\subsection{What is ECMA Script}
\subsubsection{ECMA International}
ECMA International is a not-for-profit association in Geneva. Their target is
to develop and publicate standads and technical reports for information and
communication technology and consumer electronics. 
The corresponding standard is ECMA-262 specification and ISO/IEC 16262
%quelle: http://www.ecma-international.org/activities/General/presentingecma.pdf
\subsubsection{Next standard for JavaScript}
ECMAScript 6 was standartizede by ECMA International and is the next standard
for Javascript. The actual version currently used for JavaScript is ECMAScript5.1
Browser vendors can use it as an orientation for their own JavaScript implementations.

\subsection{History}
\subsubsection*{\textbf{ES1 - 1997}}
First ECMAScript standard released by ECMA International, named ECMAScript Version 1.
\subsubsection*{\textbf{ES2 - 1998}}
One year after the release of ECMAScript Version 1, editorial changes were made to align ECMAScript with the ISO/IEC international standard 16262.
\subsubsection*{\textbf{ES3 - 1999}}
Modern language features were added in ECMAScript Version 3. Some examples are regular expression support or exception handling with try and catch.
\subsubsection*{\textbf{ES4 - (----)}}After the first set of changes made in ECMAScript Version 3, some members of ECMA International planned a set of breaking changes for Version 4, after a huge discussion about backwards compatibility Version 4 was finally abandoned.
\subsubsection*{\textbf{ES5 - 2009}}
After endless debates and discussion, the plan for breaking changes was temporarily freezzed and reduced to the clarification of JavaScript ambiguities with the introduction of the strict mode.
\subsubsection*{\textbf{ES5.1 - 2011}}
A second set of editorial changes in ECMAScript History were made to align it again with theISO/IEC standard 16262.
\subsubsection*{\textbf{ES6 - 2015}}
ECMAScript 6 (aka JS Harmony) is scheduled for june 2015. It will be the most extensive standard since ECMA Script Version 1 was released. Due to the fact, that ES6 just adds syntax features on top of ESCMAScript 5.1 the backward compatibility is guaranteed. The major changes made in ES6 also work as a foundation for the following features planned for ES7
\subsubsection*{\textbf{ES7 - (tbd)}}
At May 2015 ECMAScript Version 7 is in a very early draft state. 

\subsection{Benefits of ECMAScript6}
\subsubsection{Suitable for large code bases}
The new ECMAScript Version 6 provides modules an classes which makes the prior used patterns (e.g. module pattern) obsolete and
gives the possibility for clean easily readable code which makes ES6 suitable for bigger projects.
Thanks to the syntax improvements the boilerplate is reduced which leads to less lines of code and with proper usage to
reduced maintenance effort and easier extensibility.
\subsubsection{Straight forward migration of existing code}
As ES6 only adds new features on top of ECMAScript 5.1 the migration of existing code is very easy, wich means that no code 
must be rewritten to make the code base fit for the new standard.

\section{Web App}

\subsection{ESnake6}

\subsection{Architectural Overview}
\subsubsection{Build Process}
To build ECMAScript 6 applications while the standard is still in a draft state, we used a ES6 to ES5 compiler.
After writing the App the ECMAScript Version 6 source files will be translated into ECMAScript 5.1. while the 
used HTML and CSS files as well as the included libraries will be copied. After the translation into ECMAScript 5.1
the project can be run with RequireJS modules and Sourcemaps %TODO 
%first translate in a build step
%bild build process
\subsubsection{Libraries and Technologies}
The used technologies in the app include the Bower package manager, which was used to mange client side libraries. The task runner gulp automated the build system.
For the sake of testing http-server was used to serve static files. The RequireJS Module system handles asynchronous loading of javascript. The Loadash utility library
provides fuctions for working with lists (e.g. map, filter, last). Finally the Babel compiler compiles ECMAScript 6 into ECMAScript 5, which is understood by all modern browsers.
This specific compiler accepts a flag which causes the compiler to translate modules to RequireJS syntax.
\subsubsection{Application Strucutre}
\subsubsection{Project Structure} 
The app folder contains the nescessary HTML, CSS and JavaScript (ECMAScript6) code. The application entry point is the 

\section{ECMA Script language features}

%%%%% kommt noch
\subsection{Code Organization}
%%% kommt noch



\section{Discussion}
As it's possible to extend existing projects with ECMAScript 6 code, the question comes up if a mixture of known concepts like the Module Pattern 
and the new Module System doesn't create confusion. 

\section{Conclusion}
\begin{itemize}
        \item ECMAScript 6 adds on top of ECMAScript 5 and thus maintains backwards compatibility which makes it easy to extend existing projects with new ECMAScript 6 code. 
        \item Many syntax improvements enable the programmer to reduce boilerplate and reduce maintenance efforts. 
        \item Thanks to Modules and Classes code organization gets a lot easier and makes ECMAScript 6 suitable for bigger projects.
\end{itemize}
To use ECMAScript 6 today, you can use an ES6 to ES5 compiler e.g. Babel or Traceur. The development experience gets
very pleasant thanks to sourcemaps. For new runtime features you can also use polyfills (e.g. core-js).

\section*{Acknowledgement}
Text Text Text Text Text Text  Text Text.  \citealp{Boffelli03} might want to know about  text text text text

\paragraph{Funding\textcolon} Text Text Text Text Text Text  Text Text.

%\bibliographystyle{natbib}
%\bibliographystyle{achemnat}
%\bibliographystyle{plainnat}
%\bibliographystyle{abbrv}
%\bibliographystyle{bioinformatics}
%
%\bibliographystyle{plain}
%
%\bibliography{Document}


\begin{thebibliography}{}
\bibitem[Bofelli {\it et~al}., 2000]{Boffelli03} Bofelli,F., Name2, Name3 (2003) Article title, {\it Journal Name}, {\bf 199}, 133-154.

\bibitem[Bag {\it et~al}., 2001]{Bag01} Bag,M., Name2, Name3 (2001) Article title, {\it Journal Name}, {\bf 99}, 33-54.

\bibitem[Yoo \textit{et~al}., 2003]{Yoo03}
Yoo,M.S. \textit{et~al}. (2003) Oxidative stress regulated genes
in nigral dopaminergic neurnol cell: correlation with the known
pathology in Parkinson's disease. \textit{Brain Res. Mol. Brain
Res.}, \textbf{110}(Suppl. 1), 76--84.

\bibitem[Lehmann, 1986]{Leh86}
Lehmann,E.L. (1986) Chapter title. \textit{Book Title}. Vol.~1, 2nd edn. Springer-Verlag, New York.

\bibitem[Crenshaw and Jones, 2003]{Cre03}
Crenshaw, B.,III, and Jones, W.B.,Jr (2003) The future of clinical
cancer management: one tumor, one chip. \textit{Bioinformatics},
doi:10.1093/bioinformatics/btn000.

\bibitem[Auhtor \textit{et~al}. (2000)]{Aut00}
Auhtor,A.B. \textit{et~al}. (2000) Chapter title. In Smith, A.C.
(ed.), \textit{Book Title}, 2nd edn. Publisher, Location, Vol. 1, pp.
???--???.

\bibitem[Bardet, 1920]{Bar20}
Bardet, G. (1920) Sur un syndrome d'obesite infantile avec
polydactylie et retinite pigmentaire (contribution a l'etude des
formes cliniques de l'obesite hypophysaire). PhD Thesis, name of
institution, Paris, France.

\end{thebibliography}
\end{document}
